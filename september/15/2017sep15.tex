\documentclass[11pt]{article}

\usepackage{microtype}
\usepackage{parskip}

\usepackage{amsmath, amssymb, amsthm}
\usepackage{mathtools, thmtools}

\usepackage[T1]{fontenc}
\usepackage[ascii]{inputenc}

\usepackage[margin=60pt]{geometry}

\usepackage{fancyhdr}
\usepackage{lastpage}

\usepackage{framed}

\usepackage{bm}

\fancypagestyle{notes}{
    \fancyhf{}
    \rhead{Friday, 15 September 2017 \\}
    \chead{\thepage\ / \pageref{LastPage} \\}
    \lhead{Math 60: Multivariable Calculus \\ Taylor theorem}
    %\rhead{4 September 2017}
}

\newcommand{\uvec}[1]{\bm{\hat{#1}}}
\renewcommand{\vec}[1]{\bm{#1}}

\usepackage{tikz}
\usepackage{tikz-3dplot}
\usetikzlibrary{calc, intersections}

\pagestyle{notes}

\tdplotsetmaincoords{60}{120}

\newcommand{\real}{\mathbb R}
\newcommand{\diff}{\mathrm d}
\newcommand{\hessian}{\mathrm H}
\newcommand{\Diff}{\mathrm D}

\begin{document}

\section*{Review of one-variable Taylor stuffs}

Recall, for a one-variable function \(f\), a Taylor polynomial about a point approximates the function:
\[
    f(x) \sim P(x) = f(x_0) + f'(x_0) (x-x_0) + \frac{1}{2!} f''(x_0) (x-x_0)^2 + \dots +
        \underbrace{\frac{1}{n!} f^{(n)} (x_0) (x-x_0)^n}_\text{(remainder)}. 
\]
We want to extend this idea to multi-variable real-valued functions (recall that vector-valued functions can be considered component-wise as real-valued functions):
\[
    f(\vec x) = \underbrace{
        f(\vec {x_0}) + \nabla f(\vec {x_0}) (\vec x - \vec{x_0}) + \frac{1}{2!} (\vec x - \vec{x_0})^T H(\vec{x_0}) (\vec x - \vec{x_0})
    }_{P_2(\vec x)} + 
    \underbrace{\dots}_\text{remainder}.
\]

\paragraph{Example}

Consider the function
\[
    f(x, y) = \cos x \cos y.
\]
Find, for example, \(P_2(x, y)\) (about \((0, 0)\)).

\paragraph{Solution (method 1)}

The polynomial is
\[
    P_2(x, y) = f(0, 0) + \nabla f(0, 0) (x, y) + \frac{1}{2!} (x, y)^T \hessian f(0, 0) (x, y).
\]
Computing the various derivatives and values,
\begin{align*}
    f(0, 0) &= \cos 0 \cos 0 = 1. \\
    \nabla f &= (\partial_x f, \partial_y f) = (-\sin x \cos y, -\cos x \sin y) = (0, 0). \\
    \hessian f &=
        \begin{pmatrix}
            \partial_x \partial_x f & \partial_y \partial_x f \\
            \partial_x \partial_y f & \partial_y \partial_y f
        \end{pmatrix}
\end{align*}




\paragraph{Solution (method 2)}

We can alternatively take the Taylor polynomials of the two sub-function (\(\cos x, \cos y\)) and expand. It's an alright way to check our work.

\section*{Critical points}

Recall, in one-variable calculus, a critical point of a function \(f(x)\) is found by looking for \(c\) such that
\[
    f'(c) = 0
\]
(or undefined), and solving for various such possible \(c\).

In multi-variable, the generalization is fairly straightforward. Consider the multi-variable, real-valued function \(g(\vec x)\). The critical points \(\vec c\) are found such that
\[
    \nabla g(\vec c) = \vec 0.
\]

\paragraph{Example}

Consider 
\[
    f(x, y) = x^2 + 6y^2.
\]
Then the gradient
\[
    \nabla f(x, y) = (2x, 12 y).
\]
We set this equal to zero and thus get
\[
    \nabla f(x, y) = (2x, 12 y) = (0, 0),
\]
so that
\[
    x, y = 0
\]
necessarily.

\subsection*{Minima and maxima}

In one-variable, remember that at some critical point \(c\), if \(f''(c) > 0\), then \(c\) is a local minimum (conversely, if \(f''(c) < 0\), then \(c\) is a local max). Intuitively, we should be able to visualize



\end{document}



