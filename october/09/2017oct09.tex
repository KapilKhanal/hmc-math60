\documentclass{multi}

\title{Gauss's Divergence Theorem}
\course{Math 60: Multivariable Calculus}
\date{2017 October 9 (Monday)}

\begin{document}

\section*{Gauss's theorem}

Suppose \(W\) is a three-dimensional region bounded by \(\partial W\),
consisting of \emph{closed surfaces}, oriented so that normals point away from
\(W\), and \(\vec F\) is a vector field defined on (at least) \(W\), then
\[
\oiint_{\partial W} \vec F \cdot \diff \vec S = \iiint_W \div \vec F \, \diff
V.
\]

\paragraph{Example}

Some vector field \(\vec F\) is called \emph{incompressible} if \(\div \vec F =
0\). By Gauss's theorem,
\[
\oiint_S \vec F \cdot \diff \vec S = \iiint_W \div \vec F \, \diff V = \iiint_W 0
\, \diff V = 0.
\]
Then, the flux through any surface is zero, because the divergence is zero everywhere.

\paragraph{Example}

Let \(S\) be a closed sphere of radius \(3\).  Let \(\vec F = (y^3, x e^z, \sin
x + z)\) be a vector field. Find the flux of \(\vec F\) through \(S\).

Notice the divergence
\begin{align*}
  \div \vec F &= 0 + 0 + 1 = 1.
\end{align*}
Then the flux is given by
\begin{align*}
  \oiint \vec F \cdot \diff \vec S &= \iiint \div \vec F \, \diff V \\
  &= \div \vec F \iiint \diff V \\
  &= \iiint \diff V \\
  &= \frac 4 3 \pi r^3 \\
  &= \frac 4 3 \pi (27) \\
  &= 36 \pi.
\end{align*}

\paragraph{Example}

An \emph{inverse-square law} commonly appears in physics:
\begin{align*}
  \vec F &= \frac{q}{4 \pi \varepsilon_0} \frac{\vec r}{|\vec r|^3} \\
  &= \frac{q}{4 \pi \varepsilon_0} \frac{\uvec r}{|\vec r|^2}.
\end{align*}
Suppose we want to find the flux of such a field through a surface.  We might
try Gauss's theorem. Take the divergence:
\begin{align*}
  \div \vec F &= \frac{q}{4 \pi \varepsilon_0} \, \div \frac{(x, y,
    z)}{(x^2 + y^2 + z^2)^{\frac 3 2}} \\
  &= \frac{q}{4\pi \varepsilon_0} \left(\frac{1}{(x^2+y^2+z^2)^{\frac 3 2}} - \frac 3 2 \frac{2x^2}{(x^2+y^2+z^2)^{\frac 5 2}} + \dots\right) \\
  &= \frac{q}{4\pi \varepsilon_0} \left(\frac{3}{(x^2+y^2+z^2)^{\frac 3 2}} - \frac{3(x^2+y^2+z^2)}{(x^2+y^2+z^2)^{\frac 5 2}}\right) \\
  &= \frac{q}{4\pi \varepsilon_0} \left(\frac{3}{(x^2+y^2+z^2)^{\frac 3 2}} - \frac{3}{(x^2+y^2+z^2)^{\frac 3 2}}\right) \\
  &= 0.
\end{align*}
Then, if we apply Gauss's theorem, this tells us that the total flux through any
closed surface caused by a point charge is zero. But that can't be right! If
there's a point charge sitting somewhere, it clearly generates a non-zero
electric field outward and thus clearly generates a non-zero flux! What's with
this?

Ah! Recall that Gauss's theorem can only be applied if the field is
differentiable \emph{everywhere inside} the region bounded by the surface.  Now
notice that the divergence is actually undefined at the origin \(\vec r = (x, y,
z) = \vec 0\).  There's a sort of ``singularity'' in the field.  Then, we can't
apply Gauss's theorem.  Are we doomed?

Nope.  We can do a little trick: notice that the field generated by the charge
is spherically symmetric (\(\vec F\) does not depend on the angles \(\theta\)
and \(\phi\)).  Then, if we consider the flux through a small sphere of radius
\(a\), we simply get
\begin{gather*}
  \Phi_0 = \oiint_S \vec F \cdot \diff \vec S = \vec F \oiint_S \diff \vec S =
  |\vec F| \, 4 \pi a^2 = \frac{q}{4 \pi \epsilon_0} \frac{1}{a^2} 4 \pi a^2 = \frac{q}{\epsilon_0}.
\end{gather*}
Then, we can split the region into the small sphere centered on the point charge
and the rest, so that the flux through the whole region is given by the flux
through the sphere plus the flux through the region \emph{excluding} the small
origin sphere.  But the divergence is defined-ly zero everywhere other than the
origin, so the flux through the region excluding the origin sphere is zero, and
thus the total flux through the region is simply equal to the total flux through
the small sphere of radius \(a\):
\[
\Phi = \frac{q}{\epsilon_0}.
\]
Notice that we can always find such a sphere if the region includes the origin,
because \(a\) (which we can think of as some \(\epsilon > 0\)) can be any
arbitrary positive real.


\end{document}
