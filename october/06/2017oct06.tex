\documentclass{multi}

\title{Three Fundamental Theorems (Stokes', Green's, and Gauss's)}
\course{Math 60: Multivariable Calculus}
\date{2017 October 6 (Friday)}

\begin{document}

\section*{Stokes' Theorem}

Recall Green's theorem:
\[
    \oint_{C=\partial D} \vec F \cdot \diff \vec s = \iint_D \underbrace{(\nabla \times \vec F)}_\text{means taking some sort of derivative} \cdot \uvec z \, \diff A = \iint_D \left(\frac{\partial N}{\partial x} - \frac{\partial M}{\partial y}\right) \, \diff A,
\]
where \(\vec F = (M, N)\).

\section*{Unification}

Take some integration of a function \(\phi\):
\[
    \int \phi
\]
If that manifold is \emph{bounded} by some lower-dimensional manifold, then we make find some relationship between the integral over the boundary \(\partial M\) and the integral over the manifold \(M\):
\[
    \int_{\partial M} \phi = \int_M \underbrace{\partial \phi}_\text{some kind of derivative}.
\]

In essence, these relationships for various multi-dimensional manifolds form the ``fundamental theorems'' of calculus.

In fact, consider the one-variable fundamental theorem:
\[
    F(b) - F(a) = \int_a^b \frac{\diff}{\diff t} \, F \, \diff t.
\]
We can see this integral as an integration over the segment of the one-dimensional \(t\) curve, bounded by the endpoints \(a\) and \(b\).

\section*{Stokes' Theorem}

Consider some vector field
\[
    \vec F = (P, Q, R).
\]
We want to take this integral over a curve in \(3\)-space \(C\). We begin by considering \(C\) as the closed boundary of some surface \(S\). Then we relate the integrals over the boundary and over the surface:
\[
    \int_{C=\partial S} \vec F \cdot \diff \vec s = \iint_S \curl \vec F \cdot \diff \vec S,
\]
where \(\diff \vec s\) is the curve segment, and \(\diff \vec S\) is the surface normal element.

\subsection*{Geometric interpretations}

The ``circulation'' of \(\vec F\) along \(\partial S\) is the ``sum'' of how much field \(\vec F\) ``curls'' in \(S\).

\paragraph{Example}

Take the field \(\vec F = (-y, 2x, z)\). We find the integral over the curve 
\[
    \oint_C \vec F \cdot \diff \vec s,
\]
where \(C\) is the outwardly-oriented (counter-clockwise, looking from outside the sphere) union of three arcs along the unit sphere (at intersection with the planes):
% visualization sphere with three arcs and normal somewhere

Then the integral is
\[
    \oint_{C = C_1 \cup C_2 \cup C_3} \vec F \cdot \diff \vec s,
\]
We notice that \(C\) forms a closed curve; then, we can pick some convenient spherical surface \(S\) such that \(C\) is the boundary of \(S\). 

We calculate the curl of \(\vec F\):
\[
    \curl \vec F = 
    \begin{vmatrix}
        \uvec x & \uvec y & \uvec z \\
        \partial_x & \partial_y & \partial_z \\
        -y & 2x & z
    \end{vmatrix}
    =
    0 \uvec x + 0 \uvec y + (2 - (-1)) \uvec z
    = 3 \uvec z
    = (0, 0, 3).
\]

Then by Stokes' theorem the integral equals
\[
    \iint_S \underbrace{\curl \vec F}_{3 \uvec z} \cdot \underbrace{\diff \vec S}_{\uvec n \, \diff S}.
\]

Recall the spherical surface area element
\[
    \diff S = r^2 \sin \phi \, \diff \phi \, \diff \theta,
\]
(note \(r\) is fixed here), so that the normal element is
\[
    \diff \vec S = \uvec n \, \diff S = (\sin\phi \cos\theta, \sin \phi \sin \theta, \cos \phi).
\]


\end{document}

