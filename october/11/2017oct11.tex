\documentclass{multi}

\title{Review session}
\date{2017 October 11 (Wednesday)}
\course{Math 60: Multivariable Calculus}

\begin{document}

\section*{Exam tidbits}

\begin{enumerate}
\item
  The final exam is \emph{not} cumulative!  Instead, it will focus on
  integration (multiple integrals, \dots, Stokes', Green's, Gauss's).
\item
  Relevant homework assignments are HW7--HW12.

\item
  You are allowed a hand-written, double-sided cheat sheet.
\end{enumerate}

\section*{Miscellaneous tidbits}
% triangle of div grad curl, vector field, real fn, vector field

\subsection*{Operators and kinds of functions}

Recall the way different ``derivative-like'' operators (curl, divergence,
gradient) convert between real-valued functions (\(\real^3 \to \real\)) and
vector fields (\(\real^3 \to \real^3\)):
\begin{itemize}
\item
  The \emph{gradient} converts a real-valued function into a vector field:
  \begin{gather*}
    \nabla f(x, y, z)\colon \real^3 \to \real = \vec F(x, y, z) \colon \real^3 \to \real^3.
  \end{gather*}

\item
  The \emph{divergence} converts a vector field into a real-valued function:
  \begin{gather*}
    \nabla \cdot \vec F(x, y, z) \colon \real^3 \to \real^3 = f(x, y, z)\colon
    \real^3 \to \real.
  \end{gather*}

\item
  The \emph{curl} converts a vector field to a vector fieled:
  \begin{gather*}
    \nabla \times \vec F(x, y, z) \colon \real^3 \to \real^3 = \vec G(x, y, z)
    \colon \real^3 \to \real^3.
  \end{gather*}
  
\end{itemize}


\subsection*{Conservative curls?}

Is the curl of a vector field always conservative?  No.  Here's a
counter-example:
\begin{gather*}
  \vec F = (0, 0, x^2 + y^2).
\end{gather*}


\subsection*{Curl and conservatism}

Recall as a corollary of Green's/Stokes's theorem that a vector field \(\vec F\)
is conservative iff \(\curl F = \nabla \times \vec F = \vec 0\) on a
simply-connected regions.  What's a simply-connected region, you ask?  Loosely,
a simply-region is a region with no hole.  As far as this course goes, the only
simply-connected region we really care about is the real plane \(\real^2\).

\subsection*{What's a boundary?}

Consider a sheet of paper.  Think of the paper itself as a surface; then, the
boundary of the surface corresponds to the \emph{edge} of a paper.

Consider a bowl.  The ``surface'' of the bowl has the boundary corresponding to
the \emph{rim} of the bowl.

If a surface is closed (think of the ``skin'' of a basketball), it has no
boundary, because the surface has no \emph{edge}.  If we poke a hole in the ball
with a needle, the surface now has a boundary along the perimeter of the hole.



\section*{Homework tips}

\subsection*{Stokes's theorem}

We have to check that Stokes's theorem is true.  This is a very enlightening
homework problem, because mathematics is all about plugging things in and
checking specific examples and definitely \emph{not} about generality and
enlightening proofs.  Who needs mathematical beauty when you can do routine
computations?

Say, for example, we have a surface defined by \(x^2 + y^2 + z^2 = 2^2\) for \(z
\le 0\), a hemisphere below the \(xy\) plane.  Let the normals be oriented
outward, so that the boundary of the hemisphere should be oriented
``downward''---a \emph{clockwise} semicircle in the \(xy\) plane.

We have a vector field \(\vec F = (2y-z, x+y^2-z, 4y-3x)\).  We might like to
take the curl:
\begin{gather*}
  \curl F = ((4) - (-1), (-1) - (-3), (1) - (2)) = (5, 2, -1).
\end{gather*}


\subsection*{Applying theorems}

Suppose we have some integral over some \emph{horrible} surface \(S\):
\begin{gather*}
  \iint_S \vec F \cdot \diff \vec S.
\end{gather*}





\end{document}
