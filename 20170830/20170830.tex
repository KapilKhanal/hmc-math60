\documentclass[11pt]{article}

\usepackage{microtype}
\usepackage{parskip}

\usepackage[margin=60pt]{geometry}

\usepackage{amsmath, amssymb, amsthm}
\usepackage{mathtools, thmtools}

\usepackage[T1]{fontenc}
\usepackage[ascii]{inputenc}


\begin{document}

Single-variable calculus has two parts:
\begin{itemize}
\item \emph{Derivatives} and applications
\item \emph{Integrals} and applications
\end{itemize}
Multivariable calculus is the same, but functions are weirder.

Single variable functions map a single real variable to a single real value:
\[
    f\colon \mathbb R \to \mathbb R.
\]
\emph{Multivariable}, \emph{real-valued} functions eat multiple things and spit out a single thing:
\[
    f\colon \mathbb R^n \to \mathbb R.
\]
Single variable, \emph{multi-valued} functions eat a single thing and spit out multiple things:
\[
    f\colon \mathbb R \to \mathbb R^m.
\]
\emph{Multivariable}, \emph{multi-valued} functions east many things and spit out many things:
\[
    f\colon \mathbb R^n \to \mathbb R^m.
\]




\end{document}
